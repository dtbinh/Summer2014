\documentclass{aiaa-tc}

%\usepackage[margin=1.0in]{geometry}
\usepackage{fullpage}
\usepackage{graphicx}
\usepackage{bm} %required for bold in math mode for greek symbols
\usepackage{amsmath} %for bmatrix
\usepackage{amsfonts} %for math script font
\usepackage{url} %for website citations

\usepackage[space]{grffile} %for filepaths with spaces

\setcounter{MaxMatrixCols}{15} % for bigger bmatrices

%define degree symbol:
\newcommand{\degree}{\ensuremath{^\circ}}

\newcommand{\fr}[1]{$#1^+$} %command to write a reference frame
\newcommand{\br}[2]{[#1]_{#2}} %bracket operator with subscript
\newcommand{\tvect}[3]{\begin{bmatrix}#1\\#2\\#3\end{bmatrix}}% 3 x 1 vector
\newcommand{\tvecth}[3]{\begin{bmatrix}#1&#2&#3\end{bmatrix}}% 1 x 3 vector
\newcommand{\B}[1]{\textbf{#1}} %bold for regular vectors
\newcommand{\U}[1]{\hat{\textbf{#1}}} %hats and bold for unit vectors
\newcommand{\BG}[1]{{\bm #1}}           % for vectors using greek letters
\newcommand{\ddt}[1]{\frac{d#1}{dt}} %for time derivatives
\newcommand{\kron}{\otimes} %redefines \kron to produce kronecker product symbol, for convenience

\title{Summary of 2D cooperative estimation \\ \large{Bearing measurements only of known features}}
\author{Tim Woodbury}

\let\endtitlepage\relax %surpress line break after title page

\begin{document}

\maketitle

\section{Theoretical background and math}

The basic background describing agent dynamics and the measurement models has been described in the writeup for the case with both bearing and range measurements of landmarks. New work considers the removal of feature range measurements, while retaining knowledge of feature global positions. The estimation task is still to estimate the vehicle's inertial position, velocity, and heading angle. Egocentric feature measurements are treated exactly as before, but there are no range measurements. Similarly, the expectation of the landmark bearing derived from using another agent's measurements is a straightforward function of vehicle state estimates. The computed landmark bearing, which is treated as a measurement, is a nonlinear function of state estimates from both vehicles, interagent measurements, and the other agent's feature bearing measurement.

As previously, consider estimation executed by agent $i$. Agent $j$ makes bearing measurements $\theta_{kj}$ of feature $k$ and shares these measurements and required state estimates and covariances. As previously, the computed value $\theta_{ki}$ for the feature bearing from agent $i$ to feature $k$ is a function of the interagent position vector and the vector from $j$ to $k$:

\begin{equation}
\theta_{ki} = \theta_{ji} + \arctan{\frac{ 2\rho_{ji}\rho_{kj}\sin{(\psi_i - \psi_j + \theta_{ji} - \theta_{kj})} }{ \rho_{ji}^2 + \rho_{ki}^2 - \rho_{kj}^2 } }
\label{eq:thetaki}
\end{equation}

$\rho_{ji}$, $\theta_{ji}$, and $\theta_{kj}$ are measurements; $psi_i$ and $\psi_j$ are estimates from agent $i$ and agent $j$ respectively, as before. Now, $\rho_{kj}$ is a nonlinear function of $j$'s estimated state, rather than a measurement:

\begin{equation}
\hat{\rho}_{kj} = \| [\hat{C}_{b^j/n}]\B{r}_k - \hat{\B{r}}_j \|
\label{eq:rhokj}
\end{equation}

$[\hat{C}_{b^j/n}]$ is the direction cosine matrix from the inertial reference to agent $j$'s body reference frame, and is a function of $\hat{\psi}_j$ only. $\B{r}_k$ is known and $\B{r}_j$ is estimated.

The process for treating the egocentric measurements is entirely standard in the field of sequential state estimation. The treatment of the other agent's measurements proceeds as below:

\begin{enumerate}
\item Compute the expectation of $\theta_{ki}$ from agent $i$'s estimated state.
\item Compute the ``measured'' value of $\theta_{ki}$ from Eqs. \ref{eq:thetaki} and \ref{eq:rhokj}.
\item Compute the covariance associated with the ``measurements'' using the method of statistical linearization employed previously with measured range and bearing. In this case, the output \B{y} is a vector of length $m$, where $m$ is the number of features observed by agent $j$. \B{y} is a nonlinear function of a $6+m$ vector \B{x}, defined as $\B{x} = \begin{bmatrix}
\hat{\psi}_i & \hat{\psi}_j & \tilde{\rho}_{ji} & \tilde{\theta}_{ji} & \hat{r}_{jx} & \hat{r}_{jy} & \theta_{1j} & \theta_{2j} \dots \theta_{mj}
\end{bmatrix}^T$. The covariance matrix $R_x$ associated with \B{x} is formed from the known sensor variances and the covariance matrices of $i$ and $j$. The use of multiple estimated states from agent $j$ means that $R_x$ is not diagonal in general. The description of \B{x}, $R_x$, and Eqs. \ref{eq:thetaki}-\ref{eq:rhokj} are sufficient to compute sigma vectors and approximate the output covariance $R_y$.
\end{enumerate}

\section{Simulation results}

Preliminary simulations are conducted with accelerometer variances of $0.05$, gyroscope variance of $0.01$, and feature bearing variance of $0.01$.. Interagent range and bearing variance varies for comparison. Measurements are shared at the full rate (10 Hz). Tables \ref{tab:sig_ar10}-\ref{tab:sig_ar1} summarize errors. It appears that agent range and bearing variances of $1$ and $.01$ are near the limit at which cooperation is useful, as agent 2 performs similarly in individual and cooperative scenarios with these variances. Agent 1 demonstrates a substantial improvement in accuracy in the cooperative case. When the agent range variance is increased to 10, agent 2's accuracy drops substantially in the cooperative scenario and agent 1 derives no apparently significant benefit. 

Performance drops off very quickly when the agent bearing variance increases. Based on the performance with bearing variance of $.1$ with range variances of $1$ and $.1$, agent range measurements would need to be unreasonably precise to compensate for large agent bearing variances. In the scenario considered, $.01$ is probably near the limit for useful agent bearing variance.

\begin{table}[tb!]
\scriptsize
\centering
\begin{tabular}{c|c|c|c|c|c|c|c|c|c|c|c|}
Case & Agent & $S(\epsilon_{rix})$ & $S(\epsilon_{riy})$ & $S(\epsilon_{u})$ & $S(\epsilon_{v})$ & $S(\epsilon_{\psi})$ & $MSE(r_{ix})$ & $MSE(r_{iy})$ & $MSE(u)$ & $MSE(v)$ & $MSE(\psi)$ \\
Individual & 1& 1.21& 0.761& 0.261& 0.237& 0.0311& 1.63& 0.746& 0.0914& 0.0568& 0.0015 \\
Cooperative & 1& 1.08& 0.774& 0.266& 0.263& 0.0341& 1.52& 0.78& 0.11& 0.0694& 0.00119 \\
Individual & 2& 0.393& 0.275& 0.177& 0.164& 0.0198& 0.449& 0.132& 0.0609& 0.0276& 0.000541 \\
Cooperative & 2& 0.572& 0.435& 0.198& 0.187& 0.038& 0.659& 0.268& 0.0739& 0.0359& 0.00186 \\
\end{tabular}
\caption{Monte Carlo simulations with interagent range variance of 10 and interagent bearing variance of $.01$.}
\label{tab:sig_ar10}
\end{table}

\begin{table}[tb!]
\scriptsize
\centering
\begin{tabular}{c|c|c|c|c|c|c|c|c|c|c|c|}
Case & Agent & $S(\epsilon_{rix})$ & $S(\epsilon_{riy})$ & $S(\epsilon_{u})$ & $S(\epsilon_{v})$ & $S(\epsilon_{\psi})$ & $MSE(r_{ix})$ & $MSE(r_{iy})$ & $MSE(u)$ & $MSE(v)$ & $MSE(\psi)$ \\
Individual & 1& 1.28& 0.776& 0.266& 0.241& 0.0311& 1.81& 0.762& 0.0932& 0.0584& 0.00143 \\
Cooperative & 1& 0.739& 0.631& 0.238& 0.239& 0.0236& 0.743& 0.568& 0.0857& 0.0572& 0.000651 \\
Individual & 2& 0.385& 0.274& 0.179& 0.167& 0.0192& 0.452& 0.13& 0.0613& 0.0287& 0.00052 \\
Cooperative & 2& 0.393& 0.299& 0.172& 0.165& 0.0209& 0.548& 0.139& 0.0668& 0.0293& 0.000629 \\
\end{tabular}
\caption{Monte Carlo simulations with interagent range variance of 1 and interagent bearing variance of $.01$.}
\label{tab:sig_ar1}
\end{table}

\begin{table}[tb!]
\scriptsize
\centering
\begin{tabular}{c|c|c|c|c|c|c|c|c|c|c|c|}
Case & Agent & $S(\epsilon_{rix})$ & $S(\epsilon_{riy})$ & $S(\epsilon_{u})$ & $S(\epsilon_{v})$ & $S(\epsilon_{\psi})$ & $MSE(r_{ix})$ & $MSE(r_{iy})$ & $MSE(u)$ & $MSE(v)$ & $MSE(\psi)$ \\
Individual & 1& 1.14& 0.722& 0.256& 0.237& 0.0309& 1.44& 0.684& 0.0876& 0.0567& 0.00146 \\
Cooperative & 1& 2.22& 1.9& 0.433& 0.352& 0.0631& 5.59& 4.35& 0.268& 0.126& 0.00486 \\
Individual & 2& 0.39& 0.278& 0.175& 0.162& 0.0197& 0.448& 0.138& 0.0613& 0.0268& 0.000558 \\
Cooperative & 2& 0.946& 0.775& 0.282& 0.212& 0.0506& 1.49& 0.775& 0.142& 0.0451& 0.00302 \\
\end{tabular}
\caption{Monte Carlo simulations with interagent range variance of $1$ and interagent bearing variance of $.1$.}
\label{tab:sig_ar1_ab_.1}
\end{table}

\begin{table}[tb!]
\scriptsize
\centering
\begin{tabular}{c|c|c|c|c|c|c|c|c|c|c|c|}
Case & Agent & $S(\epsilon_{rix})$ & $S(\epsilon_{riy})$ & $S(\epsilon_{u})$ & $S(\epsilon_{v})$ & $S(\epsilon_{\psi})$ & $MSE(r_{ix})$ & $MSE(r_{iy})$ & $MSE(u)$ & $MSE(v)$ & $MSE(\psi)$ \\
Individual & 1& 1.28& 0.769& 0.266& 0.237& 0.0316& 1.82& 0.756& 0.0951& 0.057& 0.00149 \\
Cooperative & 1& 1.74& 1.53& 0.369& 0.324& 0.0552& 3.56& 2.91& 0.206& 0.106& 0.00357 \\
Individual & 2& 0.393& 0.283& 0.175& 0.163& 0.0196& 0.47& 0.139& 0.0623& 0.0274& 0.000553 \\
Cooperative & 2& 0.89& 0.718& 0.273& 0.21& 0.048& 1.37& 0.674& 0.135& 0.0448& 0.00275 \\
\end{tabular}
\caption{Monte Carlo simulations with interagent range variance of $.1$ and interagent bearing variance of $.1$.}
\label{tab:sig_ar.1_ab_.1}
\end{table}

\end{document}
